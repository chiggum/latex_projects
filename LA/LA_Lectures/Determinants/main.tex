% Copyright 2004 by Till Tantau <tantau@users.sourceforge.net>.
%
% In principle, this file can be redistributed and/or modified under
% the terms of the GNU Public License, version 2.
%
% However, this file is supposed to be a template to be modified
% for your own needs. For this reason, if you use this file as a
% template and not specifically distribute it as part of a another
% package/program, I grant the extra permission to freely copy and
% modify this file as you see fit and even to delete this copyright
% notice. 

\documentclass{beamer}



% There are many different themes available for Beamer. A comprehensive
% list with examples is given here:
% http://deic.uab.es/~iblanes/beamer_gallery/index_by_theme.html
% You can uncomment the themes below if you would like to use a different
% one:
%\usetheme{AnnArbor}
%\usetheme{Antibes}
%\usetheme{Bergen}
\usetheme{Berkeley}
%\usetheme{Berlin}
%\usetheme{Boadilla}
%\usetheme{boxes}
%\usetheme{CambridgeUS}
%\usetheme{Copenhagen}
%\usetheme{Darmstadt}
%\usetheme{default}
%\usetheme{Frankfurt}
%\usetheme{Goettingen}
%\usetheme{Hannover}
%\usetheme{Ilmenau}
%\usetheme{JuanLesPins}
%\usetheme{Luebeck}
%\usetheme{Madrid}
%\usetheme{Malmoe}
%\usetheme{Marburg}
%\usetheme{Montpellier}
%\usetheme{PaloAlto}
%\usetheme{Pittsburgh}
%\usetheme{Rochester}
%\usetheme{Singapore}
%\usetheme{Szeged}
%\usetheme{Warsaw}

\title{Determinants}

% A subtitle is optional and this may be deleted
%\subtitle{Optional Subtitle}

\author{Dhruv Kohli}
% - Give the names in the same order as the appear in the paper.
% - Use the \inst{?} command only if the authors have different
%   affiliation.

\institute[Indian Institute of Technology, Guwahati] % (optional, but mostly needed)
{
  Department of Mathematics\\
  Indian Institute of Technology, Guwahati
}
% - Use the \inst command only if there are several affiliations.
% - Keep it simple, no one is interested in your street address.

\date{}
% - Either use conference name or its abbreviation.
% - Not really informative to the audience, more for people (including
%   yourself) who are reading the slides online

\subject{Linear Algebra and its Applications}
% This is only inserted into the PDF information catalog. Can be left
% out. 

% If you have a file called "university-logo-filename.xxx", where xxx
% is a graphic format that can be processed by latex or pdflatex,
% resp., then you can add a logo as follows:

% \pgfdeclareimage[height=0.5cm]{university-logo}{university-logo-filename}
% \logo{\pgfuseimage{university-logo}}

% Delete this, if you do not want the table of contents to pop up at
% the beginning of each subsection:
\AtBeginSubsection[]
{
  \begin{frame}<beamer>{Outline}
    \tableofcontents[currentsection,currentsubsection]
  \end{frame}
}

\usepackage{biblatex}
\addbibresource{ref.bib}
\setbeamertemplate{bibliography item}{}

% Let's get started
\begin{document}
\setlength{\abovedisplayskip}{1pt}
\setlength{\belowdisplayskip}{1pt}

\begin{frame}
  \titlepage
\end{frame}

\begin{frame}{Outline}
  \tableofcontents
  % You might wish to add the option [pausesections]
\end{frame}

% Section and subsections will appear in the presentation overview
% and table of contents.
\section{Motivation}
\begin{frame}{Motivation}{}
  \begin{itemize}
  \item How to test invertibility of a matrix?
  \item How to compute volume of a box in $n$- dimensions?
  \item Any explicit formula for the solution of $Ax=b$?
  \item Any explicit formula for pivots of $A$?
  \item What is the dependence of $A^{-1}b$ on each element of $b$?
  \end{itemize}
\end{frame}

\section{Introduction}
\begin{frame}{Introduction}
\begin{enumerate}
    \item Determinant is defined only for square matrices.
    \item $detA = 0 \iff A \text{ is singular}$.
    \item $detA = $volume of a box in $n$-dimensional space.
    \item $detA = \pm (\text{product of pivots})$ where the sign depends on number of row exchanges in elimination. Even number of exchanges implies positive sign.
\end{enumerate}
\begin{itemize}
    \item \textbf{The simple things about the determinant are not the explicit formulas, but the properties it possesses.}
\end{itemize}
\end{frame}

\section{Properties of the Determinant}
\begin{frame}{Properties of the Determinant}
\begin{enumerate}
    \item $det I = 1$.
    \item Determinant changes sign when two rows are exchanged because determinant of a permutation matrix $P$ is $\pm 1$. If the number of row exchanges required to bring $P$ to $I$ is even then $detP = 1$ else $-1$.
    \item Determinant depends linearly on a row. Proof by determinant computing determinant along that row.
    \begin{align*}
        \left|\begin{matrix}a+a'&b+b'\\c&d\end{matrix}\right| &= \left|\begin{matrix}a&b\\c&d\end{matrix}\right|+\left|\begin{matrix}a'&b'\\c&d\end{matrix}\right|\\
        \left|\begin{matrix}ta&tb\\c&d\end{matrix}\right| &= t \left|\begin{matrix}a&b\\c&d\end{matrix}\right|
    \end{align*}
\end{enumerate}
\end{frame}

\begin{frame}{Properties of the Determinant}
\begin{enumerate}
    \setcounter{enumi}{3}
    \item If two rows of $A$ are equal then $detA = 0$. Proof: use $2$.
    \item Subtracting a multiple of one row from another leaves the same determinant. Proof: use $3$ and $4$.
    \item If $A$ has a zero row, then $detA = 0$. Proof: use $5$ and $4$.
    \item If $A$ is triangular then $detA = $ product of diagonal entries. Proof: use $5$ to derive diagonal matrix, then use $3$ and $1$.
    \item $detA = \pm(\text{product of pivots}), detA = 0 \iff A$ is singular. Proof: elimination leads to $U$ which has pivots on the diagonal. For singular matrices one of the row will be zero. Then use $7$.
    \item $detAB = detAdetB$. Proof: $A=P_1^TL_1U_1, B = P_2^TL_2U_2$.
    \item $detA = detA^T$. Proof: $A = P^TLU$, $A^T=U^TL^TP$ and $detP^TP = detI = 1$. This means - we can exchange rows by columns in above results.\footnote{Singular case separately for 7,8,9,10}
\end{enumerate}
\end{frame}

\section{Formulas for the Determinant}
\begin{frame}{Formulas for the Determinant}
\begin{enumerate}
    \item If $A$ is invertible then $PA=LDU$, $detP=\pm 1$ and product rule gives $detA = \pm detLdetDdetU = \pm (\text{product of pivots})$
    \item Suppose $A_{n\times n}$ is split into $n^n$ terms by applying property $3$ to each row in the following way -
    \begin{align*}
        \left|\begin{matrix}a+0&0+b\\c&d\end{matrix}\right| = \left|\begin{matrix}a&0\\c&d\end{matrix}\right|+\left|\begin{matrix}0&b\\c&d\end{matrix}\right|
    \end{align*}
    Among $n^n$ terms only $n!$ terms will be non-zero when the non-zero terms are in different columns otherwise there will be atleast one column of $0$s making determinant $0$. The $n!$ terms correspond to $n!$ permutations of $(1,\ldots,n)$ which gives another formula for determinant:
    \begin{align*}
        detA = \sum_{\text{all } P\text{'s}}a_{1\alpha}a_{2\beta}\ldots a_{n\gamma}detP
    \end{align*}
\end{enumerate}
\end{frame}

\begin{frame}{Formulas for the Determinant}
\begin{itemize}
    \item[o] Consider the terms involving $a_11$. This means $\alpha=1$. This leaves some permutation $(\beta,\ldots,\gamma)$ of resulting columns $(2,\ldots,n)$. We collect all those terms as $C_{11}$ which is the determinant of the submatrix formed by deleting row $1$ and column $1$.
    \begin{align*}
        &C_{11} = \sum_{\text{all } P\text{'s s.t. }  P_{11}=1}a_{2\beta}\ldots a_{n\gamma}detP\\
        &detA = a_{11}C_{11}+a_{12}C_{12}+\ldots+a_{1n}C_{1n}\\
        &C_{ij} = (-1)^{i+j}M_{ij}
    \end{align*}
    $M_{ij}$ is called a minor (smaller determinant) which is obtained by computing the determinant of the matrix when $i$th row and $j$th column are deleted.
\end{itemize}
\end{frame}

\section{Applications of Determinant}
\begin{frame}{Applications of Determinant}
\begin{block}{Result 1 - Computation of inverse $A^{-1}$}
$A^{-1} = C^T/detA \implies AC^{T} = detA\ I$\\
\textit{Proof - Hints}:
\begin{align*}
    (AC^{T})_{ij} = \sum_{k=1}^{n}A_{ik}C_{jk} = detA\ \mathbb{I}(i=j) 
\end{align*}
Note that when $i \neq j$, $(AC^T)_{ij}$ represents determinant of the matrix $A$ with $i$th row copied into $j$th row ($2$ rows are equal).
\end{block}
\begin{block}{Result 2 - Solution of $Ax=b$}
$x_{j} = detB_j/detA$ where $B_j$ is $A$ with $b$ in $j$th column.\\
\textit{Proof - Hints}:
\begin{align*}
    (A^{-1}b)_j = \left(\frac{C^T}{detA}b\right)_j = \frac{\sum_{k=1}^{n}C_{kj}b_k}{detA} = \frac{detB_j}{detA}
\end{align*}
\end{block}
\end{frame}

\begin{frame}{Applications of Determinant contd.}
\begin{block}{Result 3 - Volume of a box}
Volume of a box whose edges are in rows of $A$ equals $detA$.\\
\textit{Proof - Hints}: If edges are $\perp$ and are of length $l_1, l_2, \ldots, l_n$,
\begin{align*}
    AA^T = diag(l_i^2) \implies det(AA^T) = det(A)^2 = \prod_{i=1}^{n}l_i^2
\end{align*}
Sign of $detA$ will indicate whether the edges form a RH-set of coord.  $x-y-z$ or a LH-set $y-x-z$. If the edges are $\not\perp$ then with row ops. it can be made $\perp$ by reducing  matrix to  RREF. Det. is invariant to row ops, so vol. stays same.
\end{block}
\begin{block}{Result 4 - Formula of pivots}
$d_{k} = detA_k/detA_{k-1}$, $A_k$ is left submatrix of $A$ of order $k$.\\
$A_k = L_kD_kU_k \implies detA_k = \prod_{i=1}^{k}d_i \implies d_k = \frac{detA_k}{detA_{k-1}}$
\end{block}
\end{frame}

\begin{frame}{Applications of Determinant contd.}
\begin{block}{Result 5}
Elemination can be completed without row exchanges i.e. $P = I$ and $A = LU$ if and only if the leading submatrices $A_1, A_2, \ldots, A_n$ are all non-singular.\\
\textit{Proof - Hints}: Follows from result $4$.
\end{block}
\end{frame}


\section{Bibliography}
\begin{frame}[t]
    \frametitle{Bibliography}
    \nocite{*}
    \printbibliography
\end{frame}

\end{document}



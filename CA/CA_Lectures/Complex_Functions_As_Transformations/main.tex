\documentclass[12pt]{article}

\usepackage{answers}
\usepackage{setspace}
\usepackage{graphicx}
\usepackage{enumitem}
\usepackage{multicol}
\usepackage{mathrsfs}
\usepackage[margin=1in]{geometry} 
\usepackage{amsmath,amsthm,amssymb,mathtools}
\usepackage{titlesec}

\newcommand\numberthis{\addtocounter{equation}{1}\tag{\theequation}}

\titleformat{\section}[runin]{\normalfont\Large\bfseries}{\thesection}{1em}{}
\titleformat{\subsection}[runin]{\normalfont\large\bfseries}{\thesubsection}{1em}{}

\def\tf{\textbf}
\def\tt{\textit}
\def\mc{\ensuremath\mathcal}
\def\mf{\ensuremath\mathbf}
\def\mt{\ensuremath\mathit}
\def\mb{\ensuremath\mathbb}
\def\td{\ensuremath\tilde}
\def\N{\ensuremath\mb{N}}
\def\Z{\ensuremath\mb{Z}}
\def\C{\ensuremath\mb{C}}
\def\R{\ensuremath\mb{R}}
\def\S{\ensuremath\mb{S}}
\def\Ber{\ensuremath\mf{Ber}}
\def\det{\ensuremath\mf{det}}
\def\sigm{\ensuremath\mf{sigm}}
\def\diag{\ensuremath\mf{diag}}
\def\dom{\ensuremath\mf{dom}}
\def\cond{\ensuremath\mf{cond}}
\def\sign{\ensuremath\mf{sign}}
\def\bd{\ensuremath\mf{bd}}
\def\rto{\ensuremath\rightarrow\ }
\def\Rto{\ensuremath\Rightarrow\ }
\def\lto{\ensuremath\leftarrow\ }
\def\Lto{\ensuremath\Leftarrow\ }
\def\xrto{\ensuremath\xrightarrow}
\def\xRto{\ensuremath\xRightarrow}
\def\xlto{\ensuremath\xleftarrow}
\def\xLto{\ensuremath\xLeftarrow}
\def\rvec{\ensuremath\overrightarrow}
\def\lvec{\ensuremath\overleftarrow}

\providecommand\P[1]{}
\renewcommand\P[1]{\mb{P}\{#1\}}
\providecommand\E[1]{}
\renewcommand\E[1]{\mb{E}[#1]}
\providecommand\Var[1]{}
\renewcommand\Var[1]{\mf{Var}(#1)}
\providecommand\p[2]{}
\renewcommand\p[2]{\frac{\partial #1}{\partial #2}}
\providecommand\pp[2]{}
\renewcommand\pp[2]{\frac{\partial^2 #1}{\partial #2^2}}
\providecommand\ps[3]{}
\renewcommand\ps[3]{\frac{\partial^2 #1}{\partial #2\partial #3}}
\providecommand\d[2]{}
\renewcommand\d[2]{\frac{d #1}{d #2}}

\DeclareMathOperator{\sech}{sech}
\DeclareMathOperator{\csch}{csch}
    
\begin{document}
    
\title{Complex Functions as Transformations - Notes}
\author{Dhruv Kohli}
\maketitle
\section{Power Series}
\begin{itemize}
    \item If $P(z)$ is power series of $F(z)$ about $k$ then the radius of convergence of the power series is the distance of $k$ to the nearest singularity. $P(z)$ converges for all the points inside the disc and for none/some/all except singularity points on the boundary of disc and diverges outside disc.
    \item A complex power series $P(z)$ is said to converge at $a$ to the point $A$ if for all $\epsilon > 0$, there exists $N \in \N$ s.t. $|A-P_n(a)|<\epsilon$ for every value of $n$ greater than $N$. Also, if $n>m>N$ then $P_{m}(a)$ and $P_{n}(a)$ both lie within the disc, and consequently the distance between them must be less than the diameter of the disc: $|P_{n}(a)-P_{m}(a)| < 2\epsilon$. Conversely, it can be shown that if this condition is met then $P(a)$ converges.
    \item If $P(z)$ is absolutely convergent at some point, then it will also be convergent at that point. The converse is not true. Example $P(z)=\sum z^j/j$ at $-1$.
    \item If $P(z)$ converges at $z=a$, then it will also converge (in fact absolutely converge) everywhere inside the disc $|z|<|a|$.
    \item If $P(z)$ diverges at $z=d$, then it will also diverge everywhere outside the circle $|z|=|d|$.
\end{itemize}
\end{document}
        